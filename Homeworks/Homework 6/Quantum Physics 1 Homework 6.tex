\documentclass[12pt,letterpaper]{article}
\usepackage[T1]{fontenc}
\usepackage{amsmath}
\usepackage{indentfirst}
\usepackage[margin=0.5in]{geometry}
\title{Quantum Physics 1 Homework 6}
\author{Jeremy Chen}
\begin{document}
	\maketitle
	
\section*{Townsend Problem 6.1}
\subsection*{a)}
\noindent What are the energy eigenvalues for a particle of mass $m$ confined in a rectangular box with sides of length $a$, $b$, and $c$? If $a < b < c$, what is the energy of the first excited state? What is the degeneracy of this energy level?

For every single side of the rectangle, the corresponding energy eigenvalue would be different. The system would be solved for using the same technique as a cube. The formula of the energy eigenvalues of a square is the following
$$E_{n_{x}} + E_{n_{y}} + E_{n_{z}} = \frac{(n_{x}^{2} + n_{y}^{2} + n_{z}^{2})\hbar^{2}\pi^{2}}{2mL^{2}}$$
For every single axes' eigenvalue, there must be a different $L$. So the formula will be changed to the following.
$$E_{n_{x}} + E_{n_{y}} + E_{n_{z}} = \frac{\hbar^{2}\pi^{2}}{2m} \left(\frac{n_{x}^{2}}{a^{2}} + \frac{n_{y}^{2}}{b^{2}} + \frac{n_{z}^{2}}{c^{2}}\right)$$
\end{document}