\documentclass[10pt,letterpaper]{article}
\usepackage[T1]{fontenc}
\usepackage{amsmath}
\usepackage[margin=0.5in]{geometry}
\usepackage{indentfirst}
\title{Quantum Physics 1 Exam 2}
\author{Jeremy Chen}
\begin{document}
	\maketitle
\section*{Part $\alpha$ - Multiple Choices}
\subsection*{$\alpha$1)}
\noindent $\psi_{a}$ and $\psi_{b}$ denote normalized energy eigenfunctions with different eigenvalues $E_{a}$ and $E_{b}$, respectively. A system is in the state $\psi = \frac{1}{\sqrt{5}}(\psi_{a} - 2\psi_{b})$. The probability that a measurement of the energy of the system returns $E_{a}$ is:

\textbf{Answer: }D
\subsection*{$\alpha$2)}
\noindent $\psi_{a}$ and $\psi_{b}$ denote momentum eigenfunctions with different eigenvalues $a$ and $b$, respectively. What is the value of $\psi \int \psi_{a}^{*} p_{op} \psi_{b}dx$?

\textbf{Answer: }E
\subsection*{$\alpha$3)}
\noindent $\psi_{a}$ and $\psi_{b}$ denote momentum eigenfunctions with different eigenvalues $a$ and $b$, respectively. What can you say about $\psi \int \psi_{a}^{*} x_{op} \psi_{b}dx$?

\textbf{Answer: }C
\subsection*{$\alpha$4)}
\noindent Experiments to measure the rest energy of a highly unstable elementary particle give a distribution of values centered at 2500 MeV with a width of $\pm$10 MeV. Assuming the width is \textit{not} due to any defect in the measuring instrument, the order of magnitude of the lifetime of the particle is closest to which value below?

\textbf{Answer: }C
\subsection*{$\alpha$5)}
\noindent A particle in the first excited state of a one-dimensional harmonic potential well has a kinetic energy of 6.0 eV. What is the energy of this particle in the ground state? 

\textbf{Answer: }C
\subsection*{$\alpha$6}
\noindent A particle of mass $m$ is in the n=3 state in a one dimensional quantum well with potential $U(x) = \frac{1}{2}kx^{2}$. The kinetic energy of the particle is:

\textbf{Answer: }F

\section*{Part $\beta$ - Free Response}
\subsection*{$\beta$1)}
\noindent Evaluate the commutator of the operators $H_{op} = \frac{p_{op}^{2}}{2m}$ and $x_{op}$. What are the physical consequences of your calculation of the commutator? (Shared eigenstates? Uncertainty relations?...)

\textbf{Answer: }The calculation for the commutation would be to represent the operators in their formal descriptions and subtract them for the difference.
$$[H_{op}, x_{op}] = \left[-\frac{\hbar^{2}}{2m} \frac{\partial^{2}}{\partial x^{2}}, x\right]$$
Using that representation, use some arbitrary $\psi(x)$ for the commutators to act on.
$$\left( -\frac{\hbar^{2}}{2m} \frac{\partial^{2}}{\partial x^{2}}x -x\left(-\frac{\hbar^{2}}{2m} \frac{\partial^{2}}{\partial x^{2}}\right) \right) \psi$$
Distributing them into the system, it would result in 
$$-\frac{\hbar^{2}}{2m} \frac{\partial^{2}}{\partial x^{2}}(x\psi) + x\frac{\hbar^{2}}{2m} \frac{\partial^{2}\psi}{\partial x^{2}}$$
Using the product rule, the first commutation can be distributed into
$$-\frac{\hbar^{2}}{2m} \frac{\partial}{\partial x} \left( x\frac{\partial \psi}{\partial x} + \psi \right) + x\frac{\hbar^{2}}{2m} \frac{\partial^{2} \psi}{\partial x^{2}}$$
Then, by distributing the partial derivative outside of the parenthesis inside the parenthesis
$$-\frac{\hbar^{2}}{2m} \left( x\frac{\partial^{2} \psi}{\partial x^{2}} + \frac{\partial \psi}{\partial x} + \frac{\partial \psi}{\partial x} \right) + x\frac{\hbar^{2}}{2m} \frac{\partial^{2} \psi}{\partial x^{2}}$$
The second derivative in the parenthesis would cancel out with the last value outside of the parenthesis, leaving behind
$$-\frac{\hbar^{2}}{2m} \left(2\frac{\partial \psi}{\partial x}\right) = -\frac{\hbar^{2}}{m} \left(\frac{\partial \psi}{\partial x}\right)$$
By separate the coefficient into complex parts
$$\frac{\hbar}{im} \left( \frac{\hbar}{i} \frac{\partial \psi}{\partial x} \right)$$
This gets us the momentum operator, which simplifies to 
$$\frac{\hbar}{im} p_{op}$$
The physical implications of this is that the uncertainty of the two operators is proportional the the momentum of the operator. The greater the momentum of the system, the less well known both of the values will at the same time. 
\subsection*{$\beta$2)}
\noindent An operator $A$, corresponding to observable $\alpha$, has two eigenfunctions $\phi_{1}$ and $\phi_{2}$ with eigenvalues $a_{1}$ and $a_{2}$. Another operator $B$, corresponding to observable $\beta$, has two eigenfunction $\chi_{1}$ and $\chi_{2}$ with eigenvalues $b_{1}$ and $b_{2}$. The eigenfunctions are related by: 
$$\phi_{1} = (2\chi_{1} + 3\chi_{2})/\sqrt{13} \text{ and } \phi_{2} = (3\chi_{1} - 2\chi_{2})/\sqrt{13}$$
$\alpha$ is measured and the value $a_{1}$ is obtained. If $\beta$ is then measured and then $\alpha$ again, find the probability of finding $a_{1}$ again. Clearly explain your logical steps in solving this problem using words and equations.

\textbf{Answer: }The two eigenfunctions $\phi_{1}$ and $\phi_{2}$ have the following representations.
$$\begin{cases}
	\phi_{1} = \frac{2\chi_{1} + 3\chi_{2}}{\sqrt{13}} \\
	\phi_{2} = \frac{3\chi_{1} - 2\chi_{2}}{\sqrt{13}}
\end{cases}$$
since $a_{1}$ is the result we're looking for, we will look at the probability for each eigenvalue within $\phi_{1}$. The probability of getting $b_{1}$ is
$$P_{b_{1}} = \left( \frac{2}{\sqrt{13}} \right)^{2} = \frac{4}{13}$$
This can be represented in Dirac notation as $P\langle \chi_{1} | \phi_{1} \rangle$. Similarly, the probability of finding $b_{2}$ is
$$P_{b_{2}} = \left( \frac{3}{\sqrt{13}} \right)^{2} = \frac{9}{13}$$
This will be represented by $P\langle \chi_{2} | \phi_{1} \rangle$. The probability of finding $a_{1}$ then, uses the Born's postulate on both of the possible measurements. 
$$P\langle a_{1} | b_{1} \rangle + P\langle a_{1} | b_{2} \rangle = \left( \frac{4}{13} \right)^{2} + \left( \frac{9}{13} \right)^{2} = \frac{97}{169}$$
\subsection*{$\beta$3)}
\noindent A stream of electrons of energy $E > V_{1} > V_{2}$ is incident from the left onto a step-down potential at $x = 0$ as shown below. 
\subsubsection*{a)}
\noindent Write the general form of the wavefunctions for relevant spatial regions. Include the time
dependence.

\textbf{Answer: }The general form will be separated into the area before the step-down
$$Ae^{ikx - i\omega t} + Be^{-ikx - i\omega t}$$
and the area after the step-down
$$Ce^{ikx - i\omega t} + De^{-ikx - i\omega t}$$
\subsubsection*{b)}
\noindent  State the physical and boundary condition arguments for this problem. Simplify the equations for the wavefunctions using these arguments, but do not solve the problem at this point. Clearly state your assumptions and arguments.

\textbf{Answer: }The boundary condition separates the wavefunctions in terms of $x = 0$. The wavefunctions form a system of
$$\psi(x) = \begin{cases}
Ae^{ikx} + Be^{-ikx} : x < 0\\
Ce^{ikx} : x > 0
\end{cases}$$
The reason there's no wave traveling to the left for $x > 0$ is the lack of a barrier to cause a reflection. Hence, the boundary conditions are
$$A + B = C \text{    at    } x = 0$$
\subsubsection*{c)}
\noindent Solve for the mathematical relationship between transmitted probability current and incident current as a function of $E, V_{1}, V_{2}$.

\textbf{Answer: }The constant $k$ would have to be defined for both before and after the step-down. Before the step-down, the energy would be subtracted by the potential. After the step-down, it would just be the potential.
$$k = \frac{\sqrt{2m(E - V_{1})}}{\hbar},\ \ \ \ k_{0} = \frac{\sqrt{2mE}}{\hbar}$$
By using $k$ and the probability current formula, the transmitted probability current and incident current can be solved for by using the coefficients of the wavefunctions. The transmitted probability current uses the wave that got past the step-down 
$$j_{inc} = \frac{\hbar k}{m}|A|^{2}$$
and the incident probability current uses the wave that is traveling right in the well
$$j_{trans} = \frac{\hbar k_{0}}{m}|C|^{2}$$
\end{document}